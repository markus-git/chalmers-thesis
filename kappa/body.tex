%!TEX root = ../main.tex

%%%%%%%%%%%%%%%%%%%%%%%%%%%%%%%%%%%%%%%%%%%%%%%%%%%%%%%%%%%%%%%%%%%%%%%%%%%%%%%%
%
%       oe    
%     .@88    
% ==*88888    
%    88888    
%    88888    
%    88888    
%    88888    
%    88888    
%    88888    
%    88888    
% '**%%%%%%** 
%
%%%%%%%%%%%%%%%%%%%%%%%%%%%%%%%%%%%%%%%%%%%%%%%%%%%%%%%%%%%%%%%%%%%%%%%%%%%%%%%%

\chapter{Introduction}

Over the last few years, the amount of traffic going back and forth between devices in the global communications infrastructure has been increasing at a rapid pace. As the steady growth of mobile users and internet of things devices continues~\cite{ericsson2016}, the amount of mobile traffic is projected to become even greater. In fact, global internet traffic is estimated to grow at an average rate of twenty two percent annually, reaching approximately two hundred and fifty million terabytes per month by the end of 2021~\cite{cisco2016}. For the communication infrastructure, the consequence of such a rapid growth rate has been a sharp increase in the demand for computational power on systems that already run under tight latency constraints and with limited memory~\cite{persson2014}, which means computations have to be efficient.

As important as it is to increase the computational power of embedded systems used in communication, limiting their power consumption presents an equally important issue in their architectural design~\cite{mudge2001}. The trend of trading power for performance cannot continue indefinitely: the house hold processors of today typically have a power density of 70W and upwards, which is seven times that of a typical hot plate. Containing the growth in power requires architectural improvements, with specialized computing for specialized tasks. Heterogeneous computing represents an interesting development towards the goal of energy efficient computing, and refers to systems that use more than one kind of processing units. These heterogeneous systems gain their performance and energy efficiency not just by combining several processors, but rather by incorporating different kinds of co-processors that provide specialized processing capabilities to handle a particular task.

Heterogeneous computing present new challenges in software design that are not found in the development for typical homogeneous systems~\cite{kunzman2011}. The multiple processing units present in a heterogeneous system raises all of the issues associated with homogeneous parallel systems, while the heterogeneity in the system gives rise to new issues related to any dissimilarity in system development and capability. The efficiency and computational power of heterogeneous systems thus comes at a cost of increased programming burden in terms of code complexity and portability, as hardware specific code is interleaved with application code to handle any communication between co-processors. Furthermore, the structure of application code typically vary between co-processors and any code written for one therefore requires modification when given a new target. % Talk about my research.

High demands for efficiency under resource constraints have greatly influenced the development of embedded systems used for communications infrastructure. Today, digital signal processing software for embedded systems is typically written in a low level dialect of C. This choice of language is primarily driven by the desire to access the full potential of a processor or its memory system. However, low level languages also forces its developers to focus on low level implementation details rather than the high level specification of the algorithm they are implementing. This in turn has the unfortunate consequence of discouraging developers from focusing on other, less immediate aspects of software development, such as portability and modularity. Its difficult to get back these aspects later in the development cycle, as by then any design decisions will be too tightly coupled with the implementation.

% ... where the use of high level abstractions could introduce inefficiencies

This thesis consists of two parts. Part I is a general introduction to the field and puts the appended papers into context. Part II contains the appended papers.

\section{Background}

\lipsum[1]

\section{Functional programming}

\lipsum[2]

\section{Domain Specific Languages}

\lipsum[3]

\section{Domain Specific Embedded Languages}

\lipsum[4]

%%%%%%%%%%%%%%%%%%%%%%%%%%%%%%%%%%%%%%%%%%%%%%%%%%%%%%%%%%%%%%%%%%%%%%%%%%%%%%%%
%
%   .--~*teu.
%  dF     988Nx
% d888b   `8888>
% ?8888>  98888F
%  "**"  x88888~
%       d8888*`
%     z8**"`   :
%   :?.....  ..F
%  <""888888888~
%  8:  "888888*
%  ""    "**"`
%
%%%%%%%%%%%%%%%%%%%%%%%%%%%%%%%%%%%%%%%%%%%%%%%%%%%%%%%%%%%%%%%%%%%%%%%%%%%%%%%%

\chapter{Co-Design}

\lipsum[5]

\section{Section about Expression}

\lipsum[1]

\begin{code}
square :: SExp Int32 -> SExp Int32
square a = a * a
\end{code}

\lipsum[1]

\begin{stub}
square :: HExp Int32 -> HExp Int32
square a = a * a
\end{stub}

\lipsum[1]

\begin{stub}
square :: (Multiplicative exp, Type' exp a, Num a) => exp a -> exp a
square a = a * a
\end{stub}

\lipsum[1]

\begin{code}
type Point a = (a, a)

pair :: (Expr exp, Type' exp a, Num a) => Point (exp a) -> Point (exp a) -> exp a
pair (a, b) (u, v) = (a + b) * (u + v)
\end{code}

\lipsum[1]

\begin{code}
dotProd :: (Expr exp, Type' exp a, Num a) => Pull exp a -> Pull exp a -> exp a
dotProd xs ys = forLoop n 0 $ \i s -> s + xs!i * ys!i
  where
    n = min (length xs) (length ys)
\end{code}

\lipsum[1]

\begin{stub}
forLoop :: Syntax exp st => exp Length -> st -> (exp Index -> st -> st) -> st
\end{stub}

\lipsum[1]

\begin{code}
zipWith :: Expr exp => (a -> b -> c) -> Pull exp a -> Pull exp b -> Pull exp c
zipWith f xs ys = fmap (uncurry f) (zip xs ys)

sum :: (Expr exp, Type' exp a, Num a) => Pull exp a -> exp a
sum = fold (+) 0
\end{code}

\lipsum[1]

\begin{code}
scProd :: (Expr exp, Type' exp a, Num a) => Pull exp a -> Pull exp a -> exp a
scProd a b = sum (zipWith (*) a b)
\end{code}

\section{Section about Programs}

\lipsum[1]

\begin{code}
hello :: Software ()
hello = printf "Hello world!\n"
\end{code}

\lipsum[2]

\begin{code}
hello :: Software ()
hello = printf "Hello world!\n"
\end{code}

\lipsum[3]

\begin{code}
reverse :: SArr Int32 -> Software ()
reverse arr =
  do for 0 (len `div` 2) $ \ix ->
       do aix <- getArr arr ix
          ajx <- getArr arr (len - ix)
          setArr arr ix         ajx
          setArr arr (len - ix) aix
  where
    len = length arr
\end{code}

\lipsum[4]

\begin{stub}
reverse :: HArr Int32 -> Hardware ()
\end{stub}

\lipsum[5]

\begin{stub}
reverse :: (Arrays m, Expr (Exp m), Type' (Exp m) Int32) =>
  Arr m (Exp m Int32) -> m ()
\end{stub}

\lipsum[6]

%%%%%%%%%%%%%%%%%%%%%%%%%%%%%%%%%%%%%%%%%%%%%%%%%%%%%%%%%%%%%%%%%%%%%%%%%%%%%%%%
%
%   .x~~"*Weu.
%  d8Nu.  9888c
%  88888  98888
%  "***"  9888%
%       ..@8*"
%    ````"8Weu
%   ..    ?8888L
% :@88N   '8888N
% *8888~  '8888F
% '*8"`   9888%
%   `~===*%"`
%
%%%%%%%%%%%%%%%%%%%%%%%%%%%%%%%%%%%%%%%%%%%%%%%%%%%%%%%%%%%%%%%%%%%%%%%%%%%%%%%%

\chapter{Concluding Remarks}
\label{ch:conc}

\lipsum
