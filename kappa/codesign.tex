%%%%%%%%%%%%%%%%%%%%%%%%%%%%%%%%%%%%%%%%%%%%%%%%%%%%%%%%%%%%%%%%%%%%%%%%%%%%%%%%
%
%   .x~~"*Weu.
%  d8Nu.  9888c
%  88888  98888
%  "***"  9888%
%       ..@8*"
%    ````"8Weu
%   ..    ?8888L
% :@88N   '8888N
% *8888~  '8888F
% '*8"`   9888%
%   `~===*%"`
%
%%%%%%%%%%%%%%%%%%%%%%%%%%%%%%%%%%%%%%%%%%%%%%%%%%%%%%%%%%%%%%%%%%%%%%%%%%%%%%%%

\documentclass[../main.tex]{subfiles}
\begin{document}

\chapter{Co-Design}
\label{codesign}

In section~\ref{haskell} we talked about generating C code from our embedded Haskell programs, but for the kind of heterogeneous computing that our library aims to describes, only generating C is not sufficient. Typically, heterogeneous system see hardware code interleaved with application code, and even components described by the same language can support different intrinsic operations. So, starting with a single Haskell program, our co-design library is designed with three main tasks in mind: generate C code for the application parts, generate VHDL for the hardware parts, and generate a combination of C and VHDL to enable the transmission of data between software and hardware components. Furthermore, the software and hardware languages are extensible to support the addition of various intrinsic operations for different components.

C and VHDL are both imperative. To represent them, our co-design library builds on a deep embedding of monads. Monads can be thought of as composable computation descriptions, that is, they provide a means to connect smaller computations into a larger program. The general idea is that one can view an imperative program as a sequence of instructions to be executed on some machine, which looks similar to programs written in a stateful monad. In fact, a stateful program composed with monadic operations can be directly translated into statements in an imperative language. As an example, consider a software program for reversing an array:

\begin{code}
reverseS :: SArr Int32 -> Software ()
reverseS arr =
  for 0 (len `div` 2) $ \ix -> do
    aix <- getArr arr ix
    ajx <- getArr arr (len - ix)
    setArr arr ix ajx
    setArr arr (len - ix) aix
  where
    len = length arr
\end{code}

\noindent Its type tells us that it takes an array over 32-bit integers as input and produces a software program. As the array is reversed in place, the return type of the software program is simply empty, which in Haskell is show as \codei{()}. Also, the array type \codei{SArr} is prefixed with an \codei{S} to show that it is intended to be used with software statements. % an generic array type will be introduced a bit later on.

The implementation of our \codei{reverseS} function has the look and feel of a imperative program, sans a few syntactical differences. Firstly, the function performs a for-loop over the indices from zero and up to half of the arrays length. Then, for each such index \codei{ix}, it reads the array values at that index and the one from the opposite side. Lastly, these two values are put back into the array at the opposite values' index, resulting in a reversed array once the two indexed values meet in the middle. The two array operations, \codei{getArr} and \codei{setArr}, behave as you would expect them to: \codei{getArr} takes an array and an index, and returns the array's value at that index; \codei{setArr} takes an array, an index and a value, and stores the value in the array.

While the type of \codei{reverseS} is that of a software function, there's nothing software specific about it. For-loops and arrays are both part of most imperative languages, including C and VHDL. We could therefore have implemented the reverse function in hardware as well, and we do so by simple changing its type:

\begin{code}
reverseH :: HArr Int32 -> Hardware ()
reverseH arr = undefined -- same as before.
\end{code}

\noindent Other than its type, \codei{reverseH} has the same implementation as \codei{reverseS}.

Being able to implement both \codei{reverseS} and \codei{reverseH} using the same functions hints that us marking them as software and hardware specific programs was unnecessary. We would be better of keeping the reverse function generic, enabling it do be used in both languages. After all, the function only require its implementer to support for-loops and arrays to use it. The co-design library supports this kind of language agnostic functions by employing a hierarchy of type classes over operations like arrays or for-loops, with monads as the base. In the case of our reverse function, we would give it the following type:

\begin{code}
reverse :: (Monad m, Arrays m, Control m, Type m Int32) => Arr m Int32 -> m ()
reverse arr = undefined -- same as before.
\end{code}

The previous array types of \codei{SArr} and \codei{HArr} have been substituted for a more general type of \codei{Arr}, which is parameterized on the monad \codei{m}. The idea is that, a type of \codei{Arr} will turn into a \codei{SArr} or \codei{HArr} type when the monad is instantiated as \codei{Software} or \codei{Hardware}, respectively. The added constraints of \codei{Arrays} and \codei{Control} on \codei{m} ensures that whichever monad \codei{m} gets instantiated as will support the necessary array and control operations.

\begin{code}
class Monad m => Arrays m where
  type Arr m :: * -> *
  newArr  :: Type m a => Exp m Length -> m (Arr m a)
  getArr  :: Type m a => Arr m a -> Exp m Index -> m a
  setArr  :: Type m a => Arr m a -> Exp m Index -> a -> m ()

class Monad m => Control m where
  for :: (Type' m a, Integral a) => Exp m a -> Exp m a -> (Exp m a -> m ()) -> m ()
\end{code}

While \codei{Software} and \codei{Hardware} both support the above array and control operations, we can also point out groups of operations which are only supported by software but not hardware, and vice versa---like processes in hardware or IO in software. Monads still form the base of the hierarchy, but operations intended for either software or hardware branches also require that their monad is an extension of their respective \codei{Software} and \codei{Hardware} monads.

\begin{code}
class Monad m => SoftwareMonad m where
  liftSoftware :: m a -> Software a
\end{code}

% It crashes if I add the hardware monad...
% 
% class Monad m => HardwareMonad m where
%   liftHardware :: m a -> Hardware a

\section{Embedded Expressions}
\label{expr}

A language embedding based on monads gives users a representation of statements in an imperative program to work with, but most meaningful programs also include a notion of pure expressions. These expressions contain a combination of one or more values, constants, variables and operators that our library interprets and computes to produce a new value. This process is often referred to as evaluation. As an example, consider a function for squaring a value:

\begin{code}
squareS :: SExp Int32 -> SExp Int32
squareS a = a * a
\end{code}

\noindent Applying the squaring function to a value of $5$ is equaivalent to the mathematical expression $5*5$ which evaluates to $25$.

Expressions are however a deeply embedded type, and evaluation is not the only interpretation that they support. For instance, looking at the type of \codei{square} we see that its a function from a software expression over 32-bit integers to another expressions of the same type. Software expressions like this, when used in software programs, can also be compiled to C. Like the software programs from section~\ref{codesign} have a corresponding type for hardware programs, \codei{SExp} also has a corresponding hardware type for expressions called \codei{HExp}. The squaring function can be implemented using hardware expressions by simply changing its type:

\begin{stub}
squareH :: HExp Int32 -> HExp Int32
squareH a = a * a
\end{stub}

Looking at the implementation for our hardware and software squaring functions we see that they're identical; both software and hardware support a multiplication operator for their expressions. There is no software or hardware specific characteristics in the squaring function, its expression type only need to support multiplication which, in Haskell, comes from its \codei{Num} class of numerical operations:

\begin{code}
square :: (Num (exp a), Type' exp a) => exp a -> exp a
square a = a * a
\end{code}

\noindent Note that the type constraint is slightly different from previous functions, as it makes us of \codei{Type'} instead of \codei{Type}. The former type constraint is designed for functions are parameterized on their expression type, while the latter is used with monads.

In simple expressions, the resulting value is usually one of various primitive types supported by the co-design library, such as signed and unsigned numerical, floating point, and logical; in more elaborate expressions, expressions can have a composite type of smaller, andIn simple settings, the resulting value is usually one of various primitive types, such as numerical, string, and logical; in more elaborate expressions, it can be a complex data type. For instance, it is possible to make an expression over a pair of values, and we can define functions over these pairs as well:

\begin{code}
dist :: (SExp Float, SExp Float) -> (SExp Float, SExp Float) -> SExp Float
dist (x1, y1) (x2, y2) = sqrt (dx**2 + dy**2)
  where
    dx = x1 - x2
    dy = y1 - y2
\end{code}

\noindent \codei{dist} computes the distance between two points in a plane, where points are represented as a pair of coordinates. In order to compile \codei{dist}, we need to get its input from somewhere and do something with its output. A convenient way to do this in software is to use its \codei{fget} function to read standard input, \codei{stdin}, and \codei{printf} to writes the output out into standard output.

\begin{code}
distProg :: Software ()
distProg = do
  a <- fget stdin
  b <- fget stdin
  printf "dist: %d\n" (dist (a, b) (b, a))
\end{code}

Compiling \codei{distProg} produces the following code:

\begin{code}
Compile code and past result here.
\end{code}

\noindent Note that there are no pairs in the generated C code, as the different coordinates simply become four separate references to the two input variables.

\section{Communicating Components}

The ability to write software, hardware and generic programs that our co-design library means users can easily experiment with different language implementations of their functions. Most interesting heterogeneous programs does however include a mixture of software and hardware fragments, where the different parts communicate with each other. In the kind of FPGAs with embedded systems that we consider, communication is typically done over an AXI4 interconnect.

Full AXI4 offers a range of interconnects that include variable data and address bus widths, high bandwidth burst and cached transfers, and various other transaction features. The full specification of AXI4 provides great flexibility to users, and provides advanced burst and streaming transactions. For simpler programs, like the examples we have seen so far, there is no need for such advanced features. A lighter interconnect is therefore also specified by AXI4, called AXI4-lite. This subset of the full specification forgoes the more advanced features for a simpler communication model of writing and reading single pieces of data, one at a time.

Five channels make up the whole AXI4-lite specification: the read and write address channels, the read and write data channels, and the write acknowledge channel. We introduce these channels from the processor's perspective. These channels are represented in VHDL as signals, driven by processes that implement the associated handshaking and logic for reading and writing. Signals behave very much like the references found in C, and processes is a kind of function runs automatically once any of its input changes. Signals and process are both supported by our co-design library, and the whole AXI4-lite interface can in fact be implemented within the library. A function that takes a hardware program and hooks it up to an AXI4-lite interconnect is provided by the library.

% Show AXI function? Its type is quite big.

\begin{code}
axi_light :: HComp a -> HSig (
     Signal Bit       -- ^ Global clock signal.
  -> Signal Bit       -- ^ Global reset signal.
  -> Signal (Bits 32) -- ^ Write address.
  -> Signal (Bits 3)  -- ^ Write channel protection type.
  -> Signal Bit       -- ^ Write address valid.
  -> Signal Bit       -- ^ Write address ready.
  -> Signal (Bits 32) -- ^ Write data.
  -> Signal (Bits 4)  -- ^ Write strobes.
  -> Signal Bit       -- ^ Write valid.
  -> Signal Bit       -- ^ Write ready.
  -> Signal (Bits 2)  -- ^ Write response.
  -> Signal Bit       -- ^ Write response valid.
  -> Signal Bit       -- ^ Response ready.
  -> Signal (Bits 32) -- ^ Read address.
  -> Signal (Bits 3)  -- ^ Protection type.
  -> Signal Bit       -- ^ Read address valid.
  -> Signal Bit       -- ^ Read address ready.
  -> Signal (Bits 32) -- ^ Read data.
  -> Signal (Bits 2)  -- ^ Read response.
  -> Signal Bit       -- ^ Read valid.
  -> Signal Bit       -- ^ Read ready.    
  -> ())
\end{code}

There's a few new types in \codei{axi_light}. The first two types, \codei{HComp} and \codei{HSig}, are both more or less pure hardware programs, but have a type that we can inspect from within Haskell (normally, one cannot inspect the type of functions). The resulting hardware signature can then be compiled to VHDL and loaded into a synthesis tool like Vivado, which will turn it into a physical component which can be put onto the FPGA. Then, after getting the components physical address on the FPGA, we use memory-mapped I/O to access it from software:

\begin{code}
mmap :: String -> HSig a -> Software (SComp (Soften a))
\end{code}

\noindent which turns the hardware signature into a software one that we can then call as a regular function. As for the softening of \codei{a}, the hardware signature still expects signals, so those are turned into references instead.

\end{document}
