%%%%%%%%%%%%%%%%%%%%%%%%%%%%%%%%%%%%%%%%%%%%%%%%%%%%%%%%%%%%%%%%%%%%%%%%%%%%%%%%
%
%       oe    
%     .@88    
% ==*88888    
%    88888    
%    88888    
%    88888    
%    88888    
%    88888    
%    88888    
%    88888    
% '**%%%%%%** 
%
%%%%%%%%%%%%%%%%%%%%%%%%%%%%%%%%%%%%%%%%%%%%%%%%%%%%%%%%%%%%%%%%%%%%%%%%%%%%%%%%

\documentclass[../main.tex]{subfiles}
\begin{document}

\chapter{Introduction}
\label{intro}

Embedded systems are, in brief, any computer system that is part of a larger system but relies on its own microprocessor. It is embeddde as part of a larger machine to sovle a particular task, and often does so under memory and real-time constraints. Embedded systems have been have been around for decades, and still control many devices in common use today~\cite{barr2006}. For example, modern cars contain a number of embedded systems, each controlling a small function of the car; one system might control the brakes, while another displays information to the dashboard. In most cases, these embedded sub-systems are all connected together in a network. 

Developing software for embedded systems, in most cases, requires good knowledge about the hardware on which the software is supposed to run. Because embedded systems are typically designed with a task in mind, their hardware systems are often built with most cost-effective processor that meets the performance requirements. So developers must ensure their computations are efficient. Every line of code counts.

As important as computational power is for embedded systems, limiting their power consumption presents an equally important issue in their architectural design~\cite{mudge2001}; the trend of trading power for performance cannot continue indefinitely. Containing the growth in power requires architectural improvements, with specialized computing for specialized tasks.

% typical house hold processors of today have a power density of 70W and upwards, which is seven times that of a typical hot plate

% Heterogeneous computing present new challenges in software design that are not found in the development for typical homogeneous systems~\cite{kunzman2011}. The multiple processing units present in a heterogeneous system raises all of the issues associated with homogeneous parallel systems, while the heterogeneity in the system gives rise to new issues due to dissimilarity in system development and capability.

Heterogeneous computing represents an interesting development towards the goal of energy efficient computing, and refers to systems that use more than one kind of processing units. These heterogeneous systems gain their performance and energy efficiency not just by combining several processors, but rather by incorporating different kinds of co-processors that provide specialized processing capabilities to handle a particular task. Their efficiency and computational power comes at a cost of increased programming burden in terms of code complexity and portability, as hardware specific code is interleaved with software code to describe its various components and to handle any communication between co-processors.

% The efficiency and computational power of heterogeneous systems thus comes at a cost of increased programming burden in terms of code complexity and portability, as hardware specific code is interleaved with application code to handle any communication between co-processors.

% Furthermore, the structure of application code typically vary between co-processors and any code written for one therefore requires modification when given a new target. In fact, despite all the advantages heterogeneous systems offer, their use so far has been mostly restricted to hardware programmers.

A substantial amount of research has gone into addressing the challenges of programming for embedded heterogeneous systems, opening them up for programmers without a background in hardware or embedded system design. Hardware description languages are however still the most commonly used tools, together with C for specific co-processors. While these languages are good for extracting maximum performance from a processor, they provide little to no abstractions for alleviate the task of programming for heterogenous systems.

% such as VHDL and Verilog

% These hardware description languages have revolutionized hardware design but suffer from a lack of expressiveness and standardization -- there is a mismatch between description and synthesized hardware.

% Designers have therefore looked for alternative solutions, where one of the more well-known approaches is synthesis of high-level languages like C~\cite{graphics2008, ghenassia2005}. Compiling high-level languages to a hardware description has great appeal, but finding a translation between the two has however proven to be difficult; sequential programs are often a bad fit for the parallelism inherent to most hardware architectures.

Another group of languages that have shown success in describing hardware designs are functional languages~\cite{sheeran2005}. Higher-order functional languages in particular, where hardware descriptions are first-class objects, offer a particularly useful abstraction mechanism~\cite{baaij2010, bjesse1998, gill2010}. Another beneficial attribute of these languages is their purity, which enables reasoning about function (de-)composition as a seperate matter from their evaluation.

% That is, functional programming separates the definition and evaluation of its functions, and functions could, for example, be run in parallel, as long as their dependencies are covered. Functional languages are however rarely considered for embedded system development, as its difficult to give performance guarantees and resource bounds for its programs.

This thesis presents the first steps towards a functional programming language intended for embedded system development, where the entire design process of heterogeneous systems can be expressed. The language aims to go further than the aforementioned functional languages, and does not rely solely on hardware descriptions; some components are better described using sequential algorithms directly, rather than having one generated from a hardware description.

As an example, consider a dot product, also known as a scalar product. The dot product is an algebraic operation that takes two vectors of equal length and returns the sum of the products of the corresponding entries of the two vectors. In an imperative setting, like that of C, the dot product's result could be computed with a single for loop that iterates over the two input arrays and calculates the sum of their products, one pair at a time. Such a sequential solution is possible to describe in our functional language, as it provides a model of imperative programs:

\begin{code}
dot :: (MonadComp m, TypeM m Int32) => Arr m Int32 -> Arr m Int32
    -> m (Ref m Int32)
dot xs ys =
  do sum <- initRef 0
     for 0 len $ \ix ->
       do x <- getArr xs ix
          y <- getArr ys ix
          t <- getRef sum
          setRef sum (t + x * y)
     return sum
  where
    len = max (length xs) (length ys)
\end{code}

Notice that the above \codei{dot} function does not mention any software specific types; \codei{dot} only makes use of two array, a reference, a for-loop, and a few logical and numerical expressions. As a result, the function's type is kept general in the sense that its parameterized on the language \codei{m} that it will eventually be instantiated in---whether \codei{m} is a software or hardware language is not yet decided. The constraint \codei{MonadComp} on \codei{m} ensure that whichever language \codei{m} ends up as will support the necessary operations.

While the imperative implementation of the \codei{dot} function is already convenient for software realization, with the optional use as a hardware design, a pure and functional description can sometimes yield a nicer description. For this reason, our language also provide a library based on push and pull vectors~\cite{claessen2012} that facilitates a nice syntax for writing functions over vectors. For example, the dot product could be reimplemented as a function based on pull vectors:

\begin{code}
dot :: (Vector exp, Type exp Int32) => Pull exp Int32 -> Pull exp Int32
    -> exp Int32
dot a b = sum (zipWith (*) a b)
\end{code}

\noindent Which is parameterized on a pure expression language \codei{exp} rather than a stateful language, like \codei{m} in the previous incarnation of \codei{dot}.

%\begin{code}
%program :: Software ()
%program =
% do -- todo: read input arrays in a nice way.
%    r   <- dot xs ys
%    sum <- getRef ref
%    printf "sum: %d" sum
%\end{code}

%\begin{code}
%dot :: (Monad m, Arrays m, Control m, TypeM m Int32)
%  => Arr m Int32 -> Arr m Int32 -> m (Exp m Int32)
%\end{code}

%\begin{code}
%dotSig :: Signature (HArr Int32 -> HArr Int32 -> HRef Int32)
%dotSig = inpArr 16 $ \xs -> inpArr 16 $ \ys -> ret $ dot xs ys
%\end{code}

%\begin{code}
%offloaded :: Software ()
%offloaded =
%  do -- todo: read input arrays in a nice way.
%     comp <- mmap "0x83C0_0000" dotSig
%     r    <- newRef
%     call comp (xs :> ys :> r)
%     sum <- getRef r
%     printf "sum: %d" sum
%\end{code}

% This thesis consists of two parts. Part I is a general introduction to the field and puts the appended papers into context. Part II contains the appended papers.

% ToDo: Limit to FPGAs.

\end{document}
