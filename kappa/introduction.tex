%%%%%%%%%%%%%%%%%%%%%%%%%%%%%%%%%%%%%%%%%%%%%%%%%%%%%%%%%%%%%%%%%%%%%%%%%%%%%%%%
%
%       oe    
%     .@88    
% ==*88888    
%    88888    
%    88888    
%    88888    
%    88888    
%    88888    
%    88888    
%    88888    
% '**%%%%%%** 
%
%%%%%%%%%%%%%%%%%%%%%%%%%%%%%%%%%%%%%%%%%%%%%%%%%%%%%%%%%%%%%%%%%%%%%%%%%%%%%%%%

\documentclass[../paper.tex]{subfiles}
\begin{document}

\chapter{Introduction}
\label{intro}

An embedded system is, in brief, any computer system that is part of a larger system but relies on its own microprocessor. It is embedded to solve a particular task, and often does so under memory and real-time constraints, using the cheapest hardware that can meet the performance requirements. Developing for embedded systems therefore requires good knowledge about the architecture on which a program is supposed to run. Not only should computations be efficient, but they should also take full advantage of the hardware; every line of code counts.

A modern field programmable gate array is an example of an embedded system that integrates various co-processor and forms a prototypical heterogeneous computing system. The benefit of a heterogenous system like the gate array is not that several processors are combined, it is rather the incorporation of different kinds of co-processors that each one provides specialized processing capabilities to handle a particular task. A substantial amount of research has today been carried out to find good ways of programming for embedded heterogenous systems, to make them accessible for programmers without experience in embedded hardware or software system design. Hardware description languages are however still the most used tools, among with dialects of C for specific co-processors. While such low-level languages work well for extracting maximum performance from a processor, their portability is severely limited, design exploration is tedious at best, and moving entire programs between C and hardware descriptions is a major undertaking.

A group of languages that show great promise in describing both software and hardware designs are the functional languages. Higher-order functional languages in particular offer an especially useful abstraction mechanism~\cite{baaij2010, bjesse1998, gill2010} through higher-order functions and lazy evaluation. These features allow for program designs to be treated as first-class objects, and for larger applications to be constructed by composing such designs in a modular fashion; thanks to lazy evaluation, only the relevant parts of a program will show up in the generated source code. Despite the benefits of functional languages, they are rarely considered for embedded system development. One reason for this is the difficulty to give performance guarantees and resource bounds.

This thesis takes the first few steps towards a functional programming language for embedded heterogeneous systems, such as a modern field programmable gate array. Instead of taking on the full challenge of heterogeneous programming head on, a more modest approach is explored: developing a hardware software co-design language, embedding it in Haskell, and seeing how far it goes. The language is staged and utilizes the rich type system of its host language to facilitate design exploration. Furthermore, two languages, one for vector processing and one for signal processing, are introduced to accompany it. These provide useful abstractions for their respective programming domains.

As an example of the co-design language, consider a dot product (also known as a scalar product). The dot product is an algebraic operation that takes two vectors of equal length and returns the sum of all the products of corresponding entries in the two vectors:

\begin{equation}
a \cdot b = \sum_{i=1}^{n}a_{i}b_{i} = a_{1}b_{1} + a_{2}b_{2} + \cdots + a_{n}b_{n}
\end{equation}

Using an imperative language like C, the dot product's result could be computed with a single for-loop that iterates over the elements of the two arrays and calculates the sum of their products, one step at a time. Such a sequential solution can be implemented in the co-design language as well:

\begin{code}
dotSeq :: Arr Int32 -> Arr Int32 -> Program (Exp Int32)
dotSeq x y = do
  sum <- initRef 0
  for 0 (min (length x) (length y)) $ \ix -> do
    a <- getArr x ix
    b <- getArr y ix
    modifyRef sum $ \s -> s + a * b
  getRef sum
\end{code}

While the above function is faithful to its corresponding implementation in C, its low-level design has forced a focus on implementation details rather than the mathematical specification of the dot product - indices and lengths are both handled manually for instance. A more idiomatic solution is to make use of the new vector language, in which the solution can be expressed as:

\begin{code}
dotVec :: Vec Int32 -> Vec Int32 -> Program (Exp Int32)
dotVec x y = sum (zipWith (*) x y)
\end{code}

\noindent The summation and element-wise multiplication are now fully handled by the smaller \codei{sum} and \codei{zipWith} functions, respectively.

% Fix! Vector is a ... and compiles to arrays.

Vectors are supported in both C and hardware descriptions, and \codei{dotVec} can therefore be realized in either software and hardware using the co-design language. Most interesting programs for heterogeneous systems do however use of a mixture of the two. As an example, assume that \codei{dotVec} should be mapped onto hardware and that a software program then should call it and print its result to standard output. Before \codei{dotVec} can be offloaded, it needs to be given a signature over its input and output channels:

\begin{code}
comp :: Component (Arr Int32 -> Arr Int32 -> Sig Int32)
comp = inputVec 4 $ \x -> inputVec 4 $ \y -> returnVal $ dotVec x y
\end{code}

\noindent where the $4$ in \codei{inputVec} is the length of the input vectors.

A hardware component description such as \codei{comp} can automatically be connected to an AXI4-lite interconnect---a standard bus interface--- by the compiler, and generate a hardware design that is ready for synthesis. It is then possible to reach \codei{comp} from software through, for example, memory-mapped I/O. The general idea is that a memory-mapped hardware component will share its address space with the software program and can be called as a regular function:

\begin{code}
prog :: Software ()
prog = do
  dot <- mmap "0x43C00000" comp
  xs  <- initArray [1,2,3,4]
  ys  <- initArray [5,6,7,8]
  r   <- newRef
  call dot (xs .: ys .: r .: nil)
  res <- getRef r
  printf "sum: %d" res
\end{code}

\noindent where \codei{mmap} initiates the memory-mapping of \codei{comp} with its address, which is obtained during synthesis, and \codei{call} then calls the mapped component and stores its result in \codei{r}.

\end{document}
